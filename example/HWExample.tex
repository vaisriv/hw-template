\input{../lib/hw-setup/tex/HWSetup}
\input{../lib/hw-setup/tex/EngBindings}

%
% Homework Details
%   - Title
%   - Subtitle
%   - Due date
%   - Due time
%   - Course
%   - Section/Time
%   - Instructor
%   - Author
%

\hwkTitle{Title}
\hwkSubTitle{Subtitle}
\hwkDueDate{2025-01-01}
\hwkDueTime{11:59:00}
\hwkClass{Course - Section}
\hwkClassTime{09:00:00}
\hwkInstructor{Instuctor}
\hwkAuthor{Author}
\hwkCompletionDate{\today}

\begin{document}

\maketitle

\pagebreak

\begin{hwkProblem}{1}{Introducing Problems \textit{(\# pts.)}} \label{hwk:p01}

	Homework problems are placed on individual pages.

	\begin{enumerate}[label=\alph*.]
		\item \label{hwk:p01a} Instructions
		\item \label{hwk:p01b} Instructions
	\end{enumerate}

	\hwkSol{} \label{hwk:s01}

	Solutions are placed below the problem statement, and can be split

	\hwkPart{} \label{hwk:s01a}

	into

	\hwkPart{} \label{hwk:s01b}

	different parts.

\end{hwkProblem}

\begin{hwkProblem}{2}{Defining Features \textit{(\# pts.)}} \label{hwk:p02}
		
	Problems can include \( \textbf{inline math: } F_{P} = - b \dot{x}^{2} \) and \textbf{display math:}

	\[
		\Delta E_{12} = Q_{12} + W_{12} \therefore W_{12} = Q_{12}
	.\]
	
	\hwkSol{} \label{hwk:s02}

	And so can solutions. Personally, I like to use \mintinline{latex}{align*} environments (though you can use \mintinline{latex}{gather*} environments) for multi-line math.

	\begin{align*}
		m_{P} \frac{d\dot{x}}{dt}                             = F_{P}                                     \\
		m_{P} \frac{d\dot{x}}{dt}                             = -b\dot{x}^{2}                             \\
		\frac{d\dot{x}}{\dot{x}^2}                            = - \frac{b}{m_{P}} dt                      \\
		{\left[ - \frac{1}{\dot{x}} \right]}^{\dot{x}}_{v_{0}}  = - \frac{b}{m_{P}} \left( t - t_{0} \right) \\
		\dot{x}                                               = \left[ v_{0}^{-1} + \frac{b}{m_{P}} \left( t - t_{0} \right) \right] \qed
	.\end{align*}

	\hwkCode{} \label{code:s02}

	We can also display code blocks using \mintinline{latex}{minted} environments.

	\begin{minted}{python}
		minutes_to_convert = 122

		hours = int(minutes_to_convert / 60)
		minutes = minutes_to_convert % 60

		convert_label = " Minutes "
		if minutes_to_convert == 1:
		    convert_label = " Minute "

		hour_label = " Hours, "
		if hours == 1:
		    hour_label = " Hour, "

		minute_label = " Minutes"
		if minutes == 1:
		    minute_label = " Minute"

		print(
		    str(minutes_to_convert)
		    + convert_label
		    + "is the same as:\n"
		    + str(hours)
		    + hour_label
		    + str(minutes)
		    + minute_label
		)
	\end{minted}

\end{hwkProblem}

\begin{hwkProblem}{3}{File insertion \textit{(\# pts.)}} \label{hwk:p03}

	We can also insert files at will. As we can see here, the problem statement provides code for us.
	\inputminted{python}{./src/p03.py}

	\hwkSol{} \label{hwk:s03}

	I like to use the filepath convention \mintinline{bash}{./src/pXX.py} for code used in problem statements, \mintinline{bash}{./src/sXX.py} for code used in solutions, and solution program output in \mintinline{bash}{./outputs/text/sXX.txt}. (This convention can be extended to problems with multiple parts by appending the part number to the filename:\\\mintinline{bash}{./outputs/text/{s03a.py, s03b.py, etc}}.)

	\inputminted{python}{./src/s03.py}
	\inputminted{python}{./outputs/text/s03.txt}

	As you can see, our code outputs an image. I like to save them in \mintinline{bash}{./outputs/figures/sXX.png} (following the filepath naming convention we used for our code files). Let's display it here to finish off our solution. To display images, we use a combination of a \mintinline{latex}{figure} environment and a \mintinline{latex}{center} environment.

	\begin{figure}[H] \label{fig:s03}
		\begin{center}
			\includegraphics[width=0.65\textwidth]{./outputs/figures/s03.png}
		\end{center}
		\caption{Data Plot}
	\end{figure}

\end{hwkProblem}

\begin{hwkProblem}{Write number here}{Write name of problem here \textit{(\# pts.)}} \label{hwk:p04}

	Write problem statement here.

	\hwkSol{} \label{hwk:s04}

	Write solution here.

\end{hwkProblem}

\end{document}
